\section{Introducción Teórica}

Dada una matriz A $\in R^{nxn}$ queremos resolver el sistema $Ax = b$. El algoritmo de Eliminación Gaussiana se usa para simplificar el sistema de ecuaciones ya que produce una matriz triangular superior equivalente a la original. Esto facilita el despeje de las incógnitas con el uso de otro algoritmo llamado Backward Substitution.

El método Backward Substitution consiste en la obtención de los valores de las incógnitas a partir de la matriz triangulada.

El algoritmo de Factorización LU surge ante la sgte. situación: se resuelve mediante E.G. el sistema de ecuaciones $Ax = b$, pero ahora queremos resolver el mismo sistema con otro vector de términos independientes $Ax = b^{*}$. Repetir E.G. tendría un costo adicional de $O(n^{3})$ tiene como fin evitar hacer cuentas de mas, reduce la repeticion de cuentas de complejidad $O(n^{3})$ .

diagonal dominante


