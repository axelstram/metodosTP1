\subsection{Análisis de Pivoteo}

Para justificar que para el tipo de matriz banda no se produce pivoteo al aplicarse el algoritmo de Eliminación Gaussiana, analizamos propiedades varias de la matriz.
%No sabia que poner...


\subsubsection{A es diagonal dominante}
%aca va la explicacion de porque A es diagonal dominate

\subsubsection{A es no singular}
Dada la matriz A diagonal dominante, veamos que A es no singular.
Por absurdo supongamos que es singular, esto es, $\exists x\neq 0$ tal que $x \in Nu(A)$.
Como el vector es distinto a cero, existe una coordenada que es mayor al resto. Llamemosla $x_{k}$.


Del sistema de ecuaciones Ax, tomemos la ecuacion k, que tiene esta pinta:                  

$$ a_{(k, 1)} * x_{1} + a_{(k, 2)} * x_{2} +... + a_{(k, n)} * x_{n} = 0 $$\\

Luego despejamos el término k y aplicamos módulo:\\

 $$ \mid a_{(k, k)} * x_{k} \mid  =  \left \arrowvert - \sum_{j=1,j\neq k}^{n}  a_{(k,j)} * x_{j} \right \arrowvert $$

 $$ \mid a_{(k, k)}\mid * \mid x_{k} \mid \leq \sum_{j=1,j\neq k}^{n} \mid a_{(k,j)}\mid * \mid x_{j} \mid $$


 $$ \mid a_{(k, k)}\mid  = \sum_{j=1,j\neq k}^{n} \mid a_{(k,j)}\mid *  \frac{\mid x_{j} \mid}{\mid x_{k} \mid}$$\\

Como cada elemento de la sumatoria es multiplicado por $\frac{\mid x_{j} \mid}{\mid x_{k} \mid}$ $\leq 1$, disminuyen su valor, con lo que la sumatoria se reduce y resulta:\\

$$\mid a_{(k,k)} \mid < \sum_{j=1, j\neq k}^{n}\mid a_{(k,j)} \mid $$\\

Lo que es absurdo, pues A era diagonal dominante. El absurdo provino de suponer a A singular. Luego vale que A sea no singular.