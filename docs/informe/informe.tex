\documentclass[a4paper]{article}
\usepackage[spanish]{babel}
\usepackage[utf8]{inputenc}
\usepackage{graphicx}
\usepackage{enumerate}
\usepackage{listings}
\usepackage{color}
\usepackage{indentfirst}
\usepackage{fancyhdr}
\usepackage{latexsym}
\usepackage[colorlinks=true, linkcolor=black]{hyperref}
%\usepackage{makeidx}
%\usepackage{float}
\usepackage{wrapfig}
\usepackage{calc}
\usepackage{amsmath, amsthm, amssymb}
\usepackage{amsfonts}
%\lstset{language=C}
\definecolor{gray}{gray}{0.5}
\definecolor{light-gray}{gray}{0.95}
\definecolor{orange}{rgb}{1,0.5,0}

\input{page.layout}
% \setcounter{secnumdepth}{2}
\usepackage{underscore}
\usepackage{caratula}
\usepackage{url}
\usepackage{float}
\usepackage{algorithm}
\usepackage[noend]{algpseudocode}





\newcommand{\cod}[1]{{\tt #1}}
\newcommand{\negro}[1]{{\bf #1}}
\newcommand{\ital}[1]{{\em #1}}
\newcommand{\may}[1]{{\sc #1}}
\newcommand{\tab}{\hspace*{2em}}

\hypersetup{
 pdfstartview= {FitH \hypercalcbp{\paperheight-\topmargin-1in-\headheight}},
 pdfauthor={Grupo},
 pdfsubject={Dise\~{n}o}
}

\lstdefinestyle{customc}{
  backgroundcolor=\color{light-gray},
  belowcaptionskip=1\baselineskip,
  breaklines=true,
  numbers=left,
  xleftmargin=\parindent,
  language=C,
  showstringspaces=false,
  basicstyle=\footnotesize\ttfamily,
  keywordstyle=\bfseries\color{blue},
  commentstyle=\itshape\color{gray},
  identifierstyle=\color{black},
  stringstyle=\color{orange},
}

\lstdefinestyle{customasm}{
  backgroundcolor=\color{light-gray},
  belowcaptionskip=1\baselineskip,
  numbers=left,
  xleftmargin=\parindent,
  language=[x86masm]Assembler,
  keywordstyle=\bfseries\color{blue},
  basicstyle=\footnotesize\ttfamily,
  commentstyle=\itshape\color{gray},
}

\lstset{escapechar=@}


\begin{document}

\thispagestyle{empty}
\materia{Métodos Numéricos}
\submateria{Segundo Cuatrimestre de 2015}
\titulo{Trabajo Práctico I}
%\subtitulo{Scheduling}
\integrante{Yanet Giuseppin}{184/11}{yanetagiu@yahoo.com}
\integrante{Laura Muiño}{399/11}{mmuino@dc.uba.ar}
\integrante{Javier San Miguel}{786/10}{javiersm00@fmail.com}
\integrante{Axel Straminsky}{769/11}{axelstraminsky@gmail.com}

\makeatletter

\maketitle
\newpage

\thispagestyle{empty}
\vfill

\thispagestyle{empty}
\vspace{3cm}
\tableofcontents
\newpage

\newenvironment{myindentpar}[1]
{\begin{list}{1}
         {\setlength{\leftmargin}{#1}}
         \item[]
}
{\end{list} }

%\normalsize
\newpage

% -------------------------------------------------------
% Breve explicacion de la base teorica que fundamenta los metodos involucrados en el trabajo, junto con los metodos mismos.  
% -------------------------------------------------------
\subsection{Introducción Teórica}

Dada una matriz A $\in R^{nxn}$ queremos resolver el sistema $Ax = b$. El algoritmo de Eliminación Gaussiana se usa para simplificar el sistema de ecuaciones ya que produce una matriz triangular superior equivalente a la original. Esto facilita el despeje de las incógnitas con el uso de otro algoritmo llamado Backward Substitution.

El método Backward Substitution consiste en la obtención de los valores de las incógnitas a partir de la matriz triangulada.

El algoritmo de Factorización LU surge ante la sgte. situación: se resuelve mediante E.G. el sistema de ecuaciones $Ax = b$, pero ahora queremos resolver el mismo sistema con otro vector de términos independientes $Ax = b^{*}$. Repetir E.G. tendría un costo adicional de $O(n^{3})$ tiene como fin evitar hacer cuentas de mas, reduce la repeticion de cuentas de complejidad $O(n^{3})$ .

diagonal dominante




\newpage
% ------------------------------------------------------
% Análisis de los coef. de la fórmula temperatura.
% -------------------------------------------------------
\section{Desarrollo}

\subsection{Análisis de Coeficientes}
A partir de las aproximaciones de las derivadas parciales de la función $T$ distribuimos los términos para conocer los coeficientes asociados a cada punto del cual depende el valor de la temperatura que queremos conocer para un $j,k$ dado. Se debe cumplir la siguiente ecuación: \\
$$t_{(j-1, k)} (\frac{1}{\Delta^2_r}-\frac{1}{r \Delta_r}) +
t_{(j, k)} (-\frac{2}{\Delta^2_r}+\frac{1}{r \Delta_r}-\frac{2}{r^2 \Delta^2_\theta}) + 
t_{(j+1, k)} (\frac{1}{\Delta^2_r}) + 
t_{(j, k-1)} (\frac{1}{r^2 \Delta^2_\theta}) +
t_{(j, k+1)} (\frac{1}{r^2 \Delta^2_\theta}) = 0 $$ \\

Analizamos los casos en que los coeficientes obtenidos a partir de la discretización de las derivadas parciales asociados a cada punto se anulan, es decir, para que $r, \Delta_r,$ y $\Delta_\theta$ el coeficiente se anula. \\
Para el coeficiente asociado a $t_{(j-1, k)}$, $r$ debe valer $\Delta_r$. Como $r_j = (j \Delta_r) + r_i$ esto se cumple si $j$ vale 1 y $r_i$ vale cero, o $j$ vale cero y $r_i$ vale $\Delta_r$. Para el primer caso, significa que el horno no tiene radio interno, mientras que la segunda no puede suceder puesto que la ecuación la cumplen solo las temperaturas entre el radio interno y el externo (no incluidos) y si $j$ vale cero, indica que es el radio interno.
Para los coeficientes de $t_{(j+1, k)}$, $t_{(j, k-1)}$ y $t_{(j, k+1)}$, los coeficientes nunca se anulan. \\
Por último, para el coeficiente asociado a $t_{(j, k)}$, desarrollamos las sumas e igualamos a cero. \\
$$\frac{-2r^2 \Delta^2_\theta + r \Delta_r \Delta^2_\theta - 2 \Delta^2_r}{\Delta^2_r r^2 \Delta^2_\theta} = 0$$\\
El término inferior nunca se anula, por lo que debería anularse el término superior. Observamos que corresponde a una función cuadrática siendo $r$ la variable. Desarrollamos la fórmula de Bhaskara obteniendo: \\
$$r = \frac{\Delta_r}{4} \pm \frac{\sqrt{\Delta^2_r \Delta^4_\theta - 16 \Delta_\theta \Delta^2_r}}{4 \Delta^2_\theta}$$ \\
El análisis se vuelve muy complicado, pero nos valemos de saber que para que el planteo tenga sentido, $r$ debe ser mayor o igual que $\Delta_r$.
El segundo miembro es siempre positivo, por lo que si se le resta algo positivo a $\frac{\Delta_r}{4}$, seria aún menor, por lo tanto solo queda analizar el caso en que se suman los términos. Para que se cumpla la condición $\Delta_r \leq r$, queremos encontrar los $\Delta_r$ y $\Delta_\theta$ tal que: \\
$$\frac{3 \Delta_r}{4}  \leq \frac{\sqrt{\Delta^2_r \Delta^4_\theta - 16 \Delta_\theta \Delta^2_r}}{d \Delta^2_\theta} $$ \\
Para este fin, utilizamos la página web $www.wolframalpha.com$ para encontrar las soluciones al problema, las cuales indican que si $\Delta_r$ es positivo, $\Delta_\theta$ debe ser menor a cero y viceversa. Por lo tanto, el coeficiente se anula solamente para valores que no tienen sentido dentro del problema. \\
%http://www.wolframalpha.com/input/?i=sqrt%28x^2+y^4+-+16+x^2+y%29+%2F+%284*y^2%29+%3E%3D+3x%2F4
Con este análisis, concluimos que los coeficientes obtenidos de la ecuación no se anulan en los casos en que el problema planteado tiene sentido.


\subsection{Análisis de Pivoteo}

Para justificar que en la matriz banda no se produce pivoteo al aplicarse el algoritmo de Eliminación Gaussiana, procedemos a demostrar que la matriz banda es diagonal dominante. Con esto llegamos a la conclusión de que es no singular y luego....
%TODO: no se como sigue el hilo, claramente falta terminarlo. Se que si demostramos que tiene factorizacin LU entonces por una propiedad vista en clase, las submatrices ppales son no singulares (inversibles), por lo tanto podriamos aplicar el teorema ese raro que encontro axel del que no tenemos nada. Pero si no lo usamos que haciamos??

%---------------------------------------------------------------------


\subsubsection{A es diagonal dominante}
Queremos ver que A es diagonal dominante, es decir, que se cumple la siguiente inecuación:\\
$$\mid \frac{1}{\Delta^2_r}-\frac{1}{r \Delta_r}\mid +
\mid \frac{1}{\Delta^2_r} \mid + 
\mid \frac{1}{r^2 \Delta^2_\theta} \mid +
\mid \frac{1}{r^2 \Delta^2_\theta} \mid
\leq \mid -\frac{2}{\Delta^2_r}+\frac{1}{r \Delta_r}-\frac{2}{r^2 \Delta^2_\theta} \mid$$  \\
Primero, nos deshacemos de los módulos. El primer término es negativo solo cuando $r < {\Delta_r}$, pero esto no sucede debido a que $r= j \Delta_r + r_i$, con $j$ natural y $r_i$ real positivo. El resto de los términos a la izquierda de la desigualdad tambien son positivos debido a que todas las viariables estan elevadas al cuadrado, por lo tanto es equivalente no tomar módulo para estos términos. \\
Para el término de la derecha, $-\frac{2}{r^2 \Delta^2_\theta}$ siempre es negativo, por lo que para que el término sea positivo, $-\frac{2}{\Delta^2_r}+\frac{1}{r \Delta_r}$ debe de ser necesariamente positivo. Sin embargo, esto solo puede ocurrir cuando $r < \frac{\Delta_r}{2}$, por lo que siempre resulta negativo por el argumento anterior. De este modo, si multiplicamos este término por -1, obtenemos un número positivo. La inecuación resultante es: \\ 
$$ \frac{1}{\Delta^2_r}-\frac{1}{r \Delta_r} +  
\frac{1}{\Delta^2_r} + 
\frac{1}{r^2 \Delta^2_\theta} +
\frac{1}{r^2 \Delta^2_\theta}
\leq \frac{2}{\Delta^2_r}-\frac{1}{r \Delta_r}+\frac{2}{r^2 \Delta^2_\theta}$$ \\

Sumamos los términos y obtenemos: \\
$$\frac{2}{\Delta^2_r}-\frac{1}{r \Delta_r}+\frac{2}{r^2 \Delta^2_\theta} \leq \frac{2}{\Delta^2_r}-\frac{1}{r \Delta_r}+\frac{2}{r^2 \Delta^2_\theta}$$ \\
Podemos observar que se cumple la inecuación como queríamos demostrar, y en particular, vale la igualdad.


%---------------------------------------------------------------------

\subsubsection{A es no singular}
Dada la matriz A diagonal dominante, veamos que A es no singular.
Por absurdo supongamos que es singular, esto es, $\exists x\neq 0$ tal que $x \in Nu(A)$.
Como el vector es distinto a cero, existe una coordenada que es mayor al resto. Llamémosla $x_{k}$.


Del sistema de ecuaciones Ax, tomemos la ecuación k, que tiene esta forma:                  

$$ a_{(k, 1)} * x_{1} + a_{(k, 2)} * x_{2} +... + a_{(k, n)} * x_{n} = 0 $$\\

Luego despejamos el término k y aplicamos módulo:\\

 $$ \mid a_{(k, k)} * x_{k} \mid  =  \left \arrowvert - \sum_{j=1,j\neq k}^{n}  a_{(k,j)} * x_{j} \right \arrowvert $$

 $$ \mid a_{(k, k)}\mid * \mid x_{k} \mid \leq \sum_{j=1,j\neq k}^{n} \mid a_{(k,j)}\mid * \mid x_{j} \mid $$


 $$ \mid a_{(k, k)}\mid  = \sum_{j=1,j\neq k}^{n} \mid a_{(k,j)}\mid *  \frac{\mid x_{j} \mid}{\mid x_{k} \mid}$$\\

Anteriormente probamos que A es diagonal dominante y que en particular vale la igualdad. Como cada elemento de la sumatoria (que vale exactamente $a_{(k, k)}$) es multiplicado por $\frac{\mid x_{j} \mid}{\mid x_{k} \mid}$ $\leq 1$, disminuyen su valor, con lo que la sumatoria se reduce y resulta:\\

$$\mid a_{(k,k)} \mid < \sum_{j=1, j\neq k}^{n}\mid a_{(k,j)} \mid $$\\

Lo que es absurdo, pues A era diagonal dominante. El absurdo provino de suponer que A es singular. Luego vale que A es no singular.




%---------------------------------------------------------------------

\subsection{Problemas que nos encontramos durante el desarrollo}

\begin{itemize}
\item Nos encontramos con problemas a la hora de cargar la matriz A de coeficientes. Lo solucionamos realizando ejemplos de un tamaño razonable en papel, y chequeando a mano que las distintas filas estuvieran bien cargadas.

\item Problemas de precisión numérica. Inicialmente los resultados de los\ tests diferían con los resultados de la cátedra. Debido a los problemas inherentes a la aritmética de punto flotante, comenzamos experimentando con distintas maneras de realizar la sumatoria (ordenando los números de mayor a menor y de menor a mayor), ya que pensabamos que estabamos arrastrando un error. Descubrimos que las distintas maneras de realizar las sumatorias diferían en menos de $$0,00001$$, por lo que concluímos que no estabamos arrastrando errores en las cuentas. Finalmente descubrimos que el error estaba en que la variable $$m$$ en realidad era $$m+1$$, y no teníamos eso en cuenta a la hora de calcular $$\Delta r$$.

\end{itemize}



\newpage
\bibliographystyle{plain}
\bibliography{tp3}

\newpage

\begin{algorithm}
\caption{Eliminación Gaussiana}\label{euclid}
\begin{algorithmic}[1]

  \Function{GaussianElimination}{A, b}\Comment{con $A \in R^{(nxm)*(nxm)}$, $b \in R^{nxm}$}

    \State $\textit{U} = \text{A.clone()}$

    \For {$actual\_row = 0$ to $U.rows() - 2$}
      \If {$U(actual\_row, actual\_row) == 0$} 
        \Return
      \EndIf

      \For {$row = actual\_row + 1$ to $U.rows() - 1$}
          \State $coefficient = U(row, actual\_row) / U(actual\_row, actual\_row)$
          \For {$col = actual\_row$ to $U.cols() - 1$}
            \If {$col == actual\_row$}  
              \State $U(row,col) = 0$
            \Else
              \State $U(row, col) = U(row, col) - coefficient * U(actual\_row, col)$
            \EndIf
          \EndFor

          \State $b(row) = b(row) - (coefficient * b(actual\_row)$
      \EndFor    
    \EndFor

  \EndFunction

\end{algorithmic}
\end{algorithm}




\begin{algorithm}
\caption{Backward Substitution}\label{euclid}
\begin{algorithmic}[1]

  \Function{Backward Substitution}{U, b}\Comment{con $U \in R^{(nxm)*(nxm)}$, $b \in R^{nxm}$}

    \State $\textit{X} = \text{Mat(U.rows())}$

    \For {$i = U.rows() - 1$ to $0$}
      \State $acum = 0.0$

      \For {$j = i+1$ to $U.rows() - 1$} 
        \State $acum = acum + U(i, j) * X(j)$
      \EndFor

      \State $X(i) = b(i) - acum / U(i, i)$

    \EndFor
      
  \EndFunction

\end{algorithmic}
\end{algorithm}




\begin{algorithm}
\caption{Backward Substitution LU}\label{euclid}
\begin{algorithmic}[1]

  \Function{Backward Substitution LU}{LU, b}\Comment{con $LU \in R^{(nxm)*(nxm)}$, $b \in R^{nxm}$}

    \State $\textit{X} = \text{Mat(LU.rows())}$
    \State $X(0) = b(0)$

    \For {$i = 1$ to $LU.rows() - 1$}
      \State $X(i) = b(i)$

      \For {$j = 0$ to $i - 1$} 
        \State $X(i) = X(i) - LU(i,j)*X(j)$
      \EndFor

    \EndFor
      
  \EndFunction

\end{algorithmic}
\end{algorithm}




\begin{algorithm}
\caption{Generación de Matriz LU}\label{euclid}
\begin{algorithmic}[1]

  \Function{GetLU}{A, LU}\Comment{con $A \in R^{(nxm)*(nxm)}$, $LU \in R^{(nxm)*(nxm)}$}

    \State $\textit{LU} = \text{A.clone()}$

    \For {$actual\_row = 0$ to $LU.rows() - 2$}
      \If {$LU(actual\_row, actual\_row) == 0$} 
        \Return
      \EndIf

      \For {$row = actual\_row + 1$ to $LU.rows() - 1$}
          \State $coefficient = LU(row, actual\_row) / LU(actual\_row, actual\_row)$
          \For {$col = actual\_row$ to $LU.cols() - 1$}
            \If {$col == actual\_row$}  
              \State $LU(row,col) = coefficient$
            \Else
              \State $LU(row, col) = LU(row, col) - coefficient * LU(actual\_row, col)$
            \EndIf
          \EndFor

      \EndFor    
    \EndFor

  \EndFunction

\end{algorithmic}
\end{algorithm}





\begin{algorithm}
\caption{Obtención del radio de la isoterma}\label{euclid}
\begin{algorithmic}[1]

  \Function{getIsotermRadiusValues}{X, angles, isoterm, delta_r, ri}

  \Comment{con $X \in R^{nxm}$, $angles \in N$, $isoterm, delta\_r, ri \in R$}

    \State $double isotermRadious[angles]$

    \For {$i = 0$ to $angles - 1$}
      \State $isotermRadius[i] = -1$
    \EndFor

    \For {$i = 0$ to $X.rows() - angles - 1$}
      \If {$X(i) >= isoterm \wedge X(i+angles) <= isoterm$}
       \State $isotermRadius[i \% angles] = (ri + (i / angles)*delta\_r + (((X(i) - isoterm) * delta_r)/ (X(i) - X(i+angles)) ))$
      \EndIf
    \EndFor

  \EndFunction

\end{algorithmic}
\end{algorithm}

\newpage


%CHAMUYO GRAFICOS


Los casos de test fueron generados con el programa \textit{generatetest.py}. En estos, el ángulo fue fijado en $10$, y el radio ($m$) fue variando entre $5$ y $80$. Los mismos fueron corridos en una CPU Intel I5-760 de 2.80 Ghz.

Para medir el tiempo, corrimos $40$ instancias para cada $m$ y tomamos el promedio, para evitar que la presencia de outliers influyera de manera significativa en los tiempos de ejecución. No tuvimos en cuenta los tiempos de carga.

En el gráfico 1 (tiempoEGdivididoM2), lo que hicimos fue dividir el tiempo de ejecución ($O(m^{3})$) por $m^{2}$, para de esta manera obtener el gráfico de una recta y probar efectivamente que el tiempo de ejecución real del algoritmo es el dicho anteriormente.

En el gráfico 2 (tiempoLUdivididoM), lo que hicimos fue considerar 2 casos: contando el tiempo de obtención de las matrices L y U, y sin contarlo. En ambos casos dividimos el tiempo de ejecución por $m$. En el primer caso, al dividir por $m$, el gráfico crece de manera parabólica, ya que el tiempo que tarda el algoritmo en obtener las matrices L y U es $O(m^{3})$. Por lo tanto, al dividir por m, el tiempo de ejecución debería crecer de forma cuadrática, lo cual se puede observar en el gráfico. 

En el segundo caso no se tuvo en cuenta el tiempo de obtención de las matrices L y U, por lo tanto el tiempo de ejecución del algoritmo es $O(m^{2})$, y al dividir todo por $m$, obtenemos una recta, como se puede observar.

En el gráfico 3 (tiempoLUyEGsinDividir), lo que hicimos fue comparar directamente el tiempo de ejecución de la eliminación gaussiana contra el tiempo de ejecución de LU (contando el tiempo de obtención de las matrices). En este caso los tiempos de ejecución no fueron divididos por $m$ ya que solamente queríamos poder comparar a grandes rasgos las diferencias entre los tiempos de ejecución de ambos algoritmos. Por más que ambos algoritmos son $O(m^{3})$, se puede observar que la eliminación gaussiana toma considerablemente más tiempo, lo cual nos confirma empíricamente la importancia de guardarnos las matrices L y U a la hora de resolver un sistema de ecuaciones.

\input{resultados.tex}
\end{document}

