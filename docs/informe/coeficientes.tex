\documentclass{article}
\usepackage{amsmath}
\begin{document}
A partir de las aproximaciones de las derivadas parciales de la funcion $T$ distribuimos los terminos para conocer los coeficientes asociados a cada punto del cual depende el valor de la temperatura que queremos conocer para un $j,k$ dado. Se debe cumplicar la sigueinte ecuacion: \\
$t_{(j-1, k)} (\frac{1}{\Delta^2_r}-\frac{1}{r \Delta_r})$ +
$t_{(j, k)} (-\frac{2}{\Delta^2_r}+\frac{1}{r \Delta_r}-\frac{2}{r^2 \Delta^2_\theta})$ + 
$t_{(j+1, k)} (\frac{1}{\Delta^2_r})$ + \\
$t_{(j, k-1)} (\frac{1}{r^2 \Delta^2_\theta})$ +
$t_{(j, k+1)} (\frac{1}{r^2 \Delta^2_\theta})$ = 0 \\

Analizamos los casos en que los coeficientes obtenidos a partir de la discretizacion de las derivadas parciales asociados a cada punto se anulan, es decir, para que $r, \Delta_r,$ y $\Delta_\theta$ el coeficiente se anula. \\
Para el coeficiente asociado a $t_{(j-1, k)}$, $r$ debe valer $\Delta_r$. Como $r_j = (j \Delta_r) + r_i$ esto se cumple si $j$ vale 1 y $r_i$ vale cero, o $j$ vale cero y $r_i$ vale $\Delta_r$. Para el primer caso, significa que el horno no tiene radio interno, mientras que la segunda no puede suceder puesto que la ecuacion la cumplen solo las temperaturas entre el radio interno y el externo (no incluidos) y si $j$ vale cero, indica que es el radio interno.
Para los coeficientes de $t_{(j+1, k)}$, $t_{(j, k-1)}$ y $t_{(j, k+1)}$, los coeficientes nunca se anulan. \\
Por ultimo, para el coeficiente asociado a $t_{(j, k)}$, desarrollamos las sumas e igualamos a cero. \\
$\frac{-2r^2 \Delta^2_\theta + r \Delta_r \Delta^2_theta - 2 \Delta^2_r}{\Delta^2_r r^2 \Delta^2_theta} = 0$\\
El termino inferior nunca se anula, por lo que deberia anularse el termino superior. Observamos que corresponde a una funcion cuadratica siendo $r$ la variable. Desarrollamos la formula de Bhaskara obteniendo: \\
$r = \frac{\Delta_r}{4} \pm \frac{\sqrt{\Delta^2_r \Delta^4_\theta - 16 \Delta_\theta \Delta^2_r}}{4 \Delta^2_\theta}$ \\
El analisis se vuelve muy complicado, pero nos valemos de saber que para que el planteo tenga sentido, $r$ debe ser mayor o igual que $\Delta_r$.
El segundo miembro es siempre positivo, por lo que si se le resta algo positivo a $\frac{\Delta_r}{4}$, seria aun menor, por lo tanto solo queda analizar el caso en que se suman los terminos. Para que se cumpla la condicon $\Delta_r \leq r$, queremos encontrar los $\Delta_r$ y $\Delta_\theta$ tal que: \\
$\frac{3 \Delta_r}{4}  \leq \frac{\sqrt{\Delta^2_r \Delta^4_\theta - 16 \Delta_\theta \Delta^2_r}}{d \Delta^2_\theta} $ \\
Para este fin, utilizamos la pagina web $www.wolframalpha.com$ para encontrar las soluciones al problema, las cuales indican que si $\Delta_r$ es positivo, $\Delta_\theta$ debe ser menor a cero y viceversa. Por lo tanto, el coeficiente se anula solamente para valores que no tienen sentido dentro del problema. \\
%http://www.wolframalpha.com/input/?i=sqrt%28x^2+y^4+-+16+x^2+y%29+%2F+%284*y^2%29+%3E%3D+3x%2F4
Con este analisis, concluimos que los coeficientes obtenidos de la ecuacion no se anulan en los casos en que el problema planteado tiene sentido.
\end{document}