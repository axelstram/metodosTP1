\section{Conclusiones}

Como conclusión, nos resultó muy interesante la enorme diferencia temporal entre la eliminación Gaussiana y LU. Creimos en un principio que la cantidad de instancias que se iban a poder correr en un tiempo razonable con una implementacion y en la otra no, iban a ser mucho más pequeña. Realmente se ve una diferencia donde una alternativa puede resultar en que un proyecto se vuelva inciable y otra no (por ejemplo, si quisieramos implementar tal sistema en una planta real con sensores en un horno).\\

Como pendientes quedaron realizar una optimizacion del uso de la memoria con respecto a la matriz A. También nos quedo pendiente experimentar triangular junto con todos los vectores $b$ como columnas de una matriz, y resolver el sistema de ecuaciones a la vez para todos los $b$, para ver cuánto mejora el tiempo de ejecución de los algoritmos. Por el lado de los experimentos, nos hubiera gsutado contar con más tiempo para profundizar las conclusiones acerca del caso en que tenemos los valroes invertidos en un horno, variando las temperaturas exteriores y las interiores. También pensamos en hacer instancias totalmente aleatorias, pero preferimos usar casos un poco más interesantes y realistas para realizar un análisis sobre los experimentos que elegimos.  
